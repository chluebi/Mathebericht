\documentclass{article}
\usepackage[utf8]{inputenc}
\usepackage[T1]{fontenc}
\usepackage{helvet}
\renewcommand{\thesubsubsection}{\thesubsection.\alph{subsubsection}}




\title{Mathebericht - Markov Ketten}
\author{Luis und Bas}
\date{2019}

\usepackage{natbib}
\usepackage{graphicx}
\usepackage{amsmath}
\usepackage{listings}
\usepackage{graphicx}
\usepackage{sectsty}
\usepackage{comment}
\usepackage{lmodern}
\usepackage{textcomp}
\usepackage{chemformula}
\usepackage{booktabs}
\usepackage{float}
\usepackage{chemformula}

\begin{document}

\maketitle


\section{Markov Ketten: Wahrscheinlichkeit beim Tennis}
\subsection{Tennis Match 1}
\subsubsection{} Als Zusatzblatt
\subsubsection{} 
\[
M=
  \begin{pmatrix}
    0 & 0.4 & 0.6 & 0 & 0 \\
    0.6 & 0 & 0 & 0 & 0 \\
    0.4 & 0 & 0 & 0 & 0 \\
    0 & 0.6 & 0 & 1 & 0 \\
    0 & 0 & 0.4 & 0 & 1
    
  \end{pmatrix}
\]
\subsubsection{}
\[
M=
  \begin{pmatrix}
    -1 & 0.4 & 0.6 & 0 & 0 \\
    0.6 & -1 & 0 & 0 & 0 \\
    0.4 & 0 & -1 & 0 & 0 \\
    0 & 0.6 & 0 & 0 & 0 \\
    0 & 0 & 0.4 & 0 & 0
    
  \end{pmatrix}
\]

\vspace{5mm}
Gleichungen: \\
Zeile 1: a = 0.4b + 0.6c \\
Zeile 2: b = 0.6a \\
Zeile 3: c = 0.4a \\
Zeile 4: b = 0 \\
Zeile 5: c = 0 \\

\vspace{5mm}
a, b, c = 0
Gleichungsystem doppelt unterbestimmt -> Fläche

\[
w1=
  \begin{pmatrix}
    0 \\
    0 \\
    0 \\
    1 \\
    0 
    
  \end{pmatrix}
\]
\[
w2=
  \begin{pmatrix}
    0 \\
    0 \\
    0 \\
    0 \\
    1 
    
  \end{pmatrix}
\]

\subsubsection{}
\[
v=
  \begin{pmatrix}
    1 \\
    0 \\
    0 \\
    0 \\
    0 
    
  \end{pmatrix}
\]

\[
P^3 * v=
  \begin{pmatrix}
    0.00000 \\
    0.28800 \\
    0.19200 \\
    0.36000 \\
    0.16000 \\
    
  \end{pmatrix}
\]
Erklärung: \\
Dies sind die Wahrscheinlichkeiten für die verschiedenen Zustände nach 3 Schritten.

\subsubsection{}
\[
u=
  \begin{pmatrix}
    0 \\
    1 \\
    0 \\
    0 \\
    0 
    
  \end{pmatrix}
\]

\[
P^3 * u=
  \begin{pmatrix}
    0.19200 \\
    0.00000 \\
    0.00000 \\
    0.74400 \\
    0.06400 \\
    
  \end{pmatrix}
\]
\[
P^{30} * u=
  \begin{pmatrix}
    0.00000 \\
    0.00001 \\
    0.00001 \\
    0.87691 \\
    0.12307 \\
    
  \end{pmatrix}
\]
\[
P^{100} * u=
  \begin{pmatrix}
    0.00000 \\
    0.00000 \\
    0.00000 \\
    0.87692 \\
    0.12308 \\
    
  \end{pmatrix}
\]
Erklärung: \\
Bei $P^3$ ist noch ein gewisser Teil der Spiele im Einstand, während bei $P^{30}$ und $P^{100}$ (fast) alle Spiele schon beendet und meistens zugunsten von Spieler 1 ausgegangen sind.

\subsubsection{}

Lösungsweg: \\
\vspace{5mm}
\footnotesize{
\begin{lstlisting}
Funktion "f": 
function y = f (x) 
  V = [ 
   0.00000   0.00000  -0.67479  -0.00000   0.42602; 
   0.00000   0.00000   0.58438  -0.73019   0.36894; 
   0.00000   0.00000   0.38959   0.48679   0.24596; 
   1.00000   0.00000  -0.20713   0.43811  -0.72064; 
   0.00000   1.00000  -0.09206  -0.19472  -0.32028; 
  ] 
  goal = [1; 0; 0; 0; 0] 
  y = zeros(5,1) 
  y(1) = x(1) * V(1,1) + x(2) * V(1,2) + x(3) * V(1,3) + x(4) * V(1,4) + x(5) * V(1,5) - 1 
  y(2) = x(1) * V(2,1) + x(2) * V(2,2) + x(3) * V(2,3) + x(4) * V(2,4) + x(5) * V(2,5) 
  y(3) = x(1) * V(3,1) + x(2) * V(3,2) + x(3) * V(3,3) + x(4) * V(3,4) + x(5) * V(3,5) 
  y(4) = x(1) * V(4,1) + x(2) * V(4,2) + x(3) * V(4,3) + x(4) * V(4,4) + x(5) * V(4,5) 
  y(5) = x(1) * V(5,1) + x(2) * V(5,2) + x(3) * V(5,3) + x(4) * V(5,4) + x(5) * V(5,5) 
endfunction
\end{lstlisting}
}
\begin{lstlisting}
Datei "tennis1": 
P = [0, 0.4, 0.6, 0, 0; 
    0.6, 0, 0, 0, 0; 
    0.4, 0, 0, 0, 0; 
    0, 0.6, 0, 1, 0; 
    0, 0, 0.4, 0, 1] 
[x, fval, info] = fsolve (@f, [1; 1; 1; 1; 1]) 
\end{lstlisting}
\vspace{5mm}
\normalsize
Output von "tennis1":
\[
x=
  \begin{pmatrix}
    0.6923066298 \\
    0.3076859606 \\
    -0.7409688692 \\
    0.0000025369 \\
    1.1736576142 \\
    
  \end{pmatrix}
\]

\subsubsection{}

$\lim\limits_{n \rightarrow \infty}{M^n} = M^n * v =$ \[
  \begin{pmatrix}
    z1 \\
    z2 \\
    z3 \\
    z4 \\
    z5 \\
  \end{pmatrix}
\]
\\
z4 gesucht! \\ \\
\vspace{5mm}
$\lim\limits_{n \rightarrow \infty}{M^n * v} = M^n * x_1 * w_1 + M^n * x_2 * w_2 ... + M^n * x_5 * w_5 \\
= \lambda_1 * x_1 * w_1 + \lambda_2 * x_2 * w_2 ... + \lambda_5* x_5 * w_5 \\$
\vspace{5mm} \\
Weil: $\lambda_1 = \lambda_2 = 0$ und  $\lambda_3 < 0$, $\lambda_4 < 0$, $\lambda_5 < 0$\\
\vspace{5mm} \\
$\lim\limits_{n \rightarrow \infty}{M^n * v} = x_1 * w_1  + x_2 * w_2 = endvektor$ \\
 \\
$endvektor_4 = 0.69231$


\subsubsection{}
$P^{10000} * v$ = \[
  \begin{pmatrix}
    0 \\
    0 \\
    0 \\
    0.69231 \\
    0.30769 \\
    
  \end{pmatrix}
\]

Chance $= 69.231 \%$


\subsection{Tennis Match 2}
\subsubsection{}
\begin{lstlisting}
function P = prob_win (p)
  P = [0, 1-p, p, 0, 0;
      p, 0, 0, 0, 0;
      1-p, 0, 0, 0, 0;
      0, p, 0, 1, 0;
      0, 0, 1-p, 0, 1]
endfunction
\end{lstlisting}

\subsubsection{}
\begin{lstlisting}
function [P, lincomb] = prob_win (p, v)
  P = [0, 1-p, p, 0, 0;
      p, 0, 0, 0, 0;
      1-p, 0, 0, 0, 0;
      0, p, 0, 1, 0;
      0, 0, 1-p, 0, 1];
  
  [w, LAMBDA] = eig(P);
  
  lincomb = w \ v;
endfunction
\end{lstlisting}

\subsubsection{}
\begin{lstlisting}
function [P, lincomb, win_chance] = prob_win (p, v)
  P = [0, 1-p, p, 0, 0;
      p, 0, 0, 0, 0;
      1-p, 0, 0, 0, 0;
      0, p, 0, 1, 0;
      0, 0, 1-p, 0, 1];
  
  [w, LAMBDA] = eig(P);
  
  lincomb = w \ v;
  
  win_chance = lincomb(1);
endfunction
\end{lstlisting}

\subsubsection{}
Kleines Umschreiben von "prob\_win.m": \\
\begin{lstlisting}
function [win_chance, lincomb, P] = prob_win (p, v)
  P = [0, 1-p, p, 0, 0;
      p, 0, 0, 0, 0;
      1-p, 0, 0, 0, 0;
      0, p, 0, 1, 0;
      0, 0, 1-p, 0, 1];
  
  [w, LAMBDA] = eig(P);
  
  lincomb = w \ v;
  
  win_chance = lincomb(1);
endfunction
\end{lstlisting}

tennis2.m: \\
\begin{lstlisting}
v = eye(5,1);
k = [];
o = [];
for i=1:100
	win_chance = prob_win((i-1)/100, v);
	k(i) = win_chance;
  o(i) = (i-1)/100;
endfor
plot(o, k)
\end{lstlisting}

\begin{figure}[h!]
\centering
\includegraphics[scale=0.4]{graph.png}
\caption{Graph von tennis2.m}
\label{fig:universe}
\end{figure}

\subsubsection{}
tennis2\_alle.m: \\
\begin{lstlisting}
hold on;

v = eye(5,1);
k = [];
o = [];
for i=1:100
	win_chance = prob_win((i-1)/100, v);
	k(i) = win_chance;
  o(i) = (i-1)/100;
endfor
plot(o, k)

v = [0;1;0;0;0];
k = [];
o = [];
for i=1:100
	win_chance = prob_win((i-1)/100, v);
	k(i) = win_chance;
  o(i) = (i-1)/100;
endfor
plot(o, k)

v = [0;0;1;0;0];
k = [];
o = [];
for i=1:100
	win_chance = prob_win((i-1)/100, v);
	k(i) = win_chance;
  o(i) = (i-1)/100;
endfor
plot(o, k)

\end{lstlisting}

\begin{figure}[h!]
\centering
\includegraphics[scale=0.4]{graph2.png}
\caption{Graph von tennis2\_alle.m}
\label{fig:universe}
\end{figure}

\subsection{Tennis Match 3}

\subsubsection{}
Chance auf Einstand:

1 - (Chance auf Sieg + Chance auf Niederlage)

$= 1 - (p^2 + (1-p)^2)$

$= 1 - (p^2 + 1 - 2p + p^2)$

$= 1 - 2p^2 - 1 + 2p$

$= 2p - 2p^2$

\subsubsection{}
Chance auf Sieg nach drei weitere Einstände:

$($Chance auf Einstand$)^3 * ($Chance auf Sieg$)$

$= (2p - 2p^2)^3 * p^2$

\subsubsection{}
Berechnung der Siegeschance:

$$\lim_{n\to\infty} \sum_{k=0}^{n} p^2 * (2p-2p^2)^k$$

Geometrische Folge:

\[=\frac{p^2}{1-(2p-2p^2)}\]

\[=\frac{0.6^2}{1-(2*0.6-2*(0.6^2))}\]

\[=\frac{0.36}{1-(1.2-0.72)}\]

\[=0.6923076923\]

\subsubsection{}
%TODO: ADD PICTURE & CODE

\section{Populationsmodelle}
\subsubsection{}
Auf Zusatzblatt
\subsubsection{}
Bleibt konstant \\
Erklärung: Es kommen immer 6 dazu, danach durch 2 und durch 3 -> wieder bei 1
\subsubsection{}
eig(A1) = 2, -1, -1 -> wird immer grösser, dominanter (=absolut gesehen grösster) > 1 \\
eig(A2) = 0.75, 0.375, 0.375 -> wird immer kleiner, dominanter (=absolut gesehen grösster) < 1 \\
eig(A3) = 1, -0.5, -0.5 -> bleibt konstant, dominanter (=absolut gesehen grösster) = 1 \\
%eig(A4) = 1, -1 -> bleibt gleich \\
\subsubsection{}
plot\_pop\_sum.m:
\begin{lstlisting}
function endmatrix = plot_pop_sum (M)
  endmatrix = zeros(1,11);
  for i=0:10
    v = ones(size(M)(1),1) * 10;
    endmatrix(1,i+1) = sum(M^i * v);
  endfor
  endmatrix
  plot(0:10,endmatrix)
endfunction
\end{lstlisting}

\begin{figure}[H]
\centering
\includegraphics[scale=0.6]{plotA1.png}
\caption{plot\_pop\_sum(A1)}
\label{fig:universe}
\end{figure}

\begin{figure}[H]
\centering
\includegraphics[scale=0.5]{plotA2.png}
\caption{plot\_pop\_sum(A2)}
\label{fig:universe}
\end{figure}

\begin{figure}[H]
\centering
\includegraphics[scale=0.5]{plotA3.png}
\caption{plot\_pop\_sum(A3)}
\label{fig:universe}
\end{figure}

\begin{figure}[H]
\centering
\includegraphics[scale=0.5]{plotA4.png}
\caption{plot\_pop\_sum(A4)}
\label{fig:universe}
\end{figure}

\begin{figure}[H]
\centering
\includegraphics[scale=0.5]{plotA5.png}
\caption{plot\_pop\_sum(A5)}
\label{fig:universe}
\end{figure}

\subsubsection{}
plot\_pop\_rel.m:
\begin{lstlisting}
function endmatrix = plot_pop_rel (M)
  endmatrix = zeros(size(M)(1),11);
  for i=0:10
    v = ones(size(M)(1),1) * 10;
    res = M^i * v;
    endmatrix(:,i+1) = res / sum(res);
  endfor
  endmatrix
  plot(0:10,endmatrix(1,:), 0:10, endmatrix(2,:), 0:10, endmatrix(3,:))
endfunction
\end{lstlisting}

\begin{figure}[H]
\centering
\includegraphics[scale=0.5]{plotrelA1.png}
\caption{plot\_pop\_rel(A1)}
\label{fig:universe}
\end{figure}

\begin{figure}[H]
\centering
\includegraphics[scale=0.5]{plotrelA2.png}
\caption{plot\_pop\_rel(A2)}
\label{fig:universe}
\end{figure}

\begin{figure}[H]
\centering
\includegraphics[scale=0.5]{plotrelA3.png}
\caption{plot\_pop\_rel(A3)}
\label{fig:universe}
\end{figure}

\subsubsection{}
Am Beispiel A1: \\
[V, LAM] = eig(A1)
\[
V =
  \begin{pmatrix}
    -0.96935 + 0.00000i && 0.88465 + 0.00000i && 0.88465 - 0.00000i \\ 
    -0.24234 + 0.00000i && -0.44233 - 0.00000i && -0.44233 + 0.00000i \\
    -0.04039 + 0.00000i &&  0.14744 + 0.00000i && 0.14744 - 0.00000i \\
  \end{pmatrix}
\]
\[
LAM =
  \begin{pmatrix}
    2.00000 + 0.00000i && 0 && 0 \\ 
    0 && -1.00000 + 0.00000i && 0 \\
    0 &&  0 &&  -1.00000 - 0.00000i \\
  \end{pmatrix}
\]

Der Eigenwert \textbf{2} ist der dominate Eigenwert und wird langfristig gesehen der einzig wichtige Eigenvektor sein. Dessen dazugehöriger Eigenvektor zeigt deshalb auch die Aufteilung der Population langfristig gesehen.

\subsubsection{}
Die Eigenwerte von A4 = [1, -1] und A5 = [-0.5 + 0.86603i, -0.5 + 0.86603i, 1] haben alle den gleichen Betrag (=1). Es gibt keinen dominaten Eigenwert und man muss alle berücksichtigen.

\subsubsection{}
Sage man habe die Matrix M und deren Eigenwerte $\lambda_1, \lambda_2 ... \lambda_n$ \\
Man berechne nun für jeden Eigenwert $\lambda_k$ den Betrag $|\lambda_k|$. Gibt es einen Eigenwert welcher grösser ist als alle anderen so ist nur dieser wichtig. Sind die Beträge mehrer Eigenwerte gleich gross, so sind alle dominant und müssen berücksichtigt werden.

\vspace{5mm}

$Wachstum = \left\{
\begin{array}{ll}
|\lambda_{dom}| > 1 & \, \textrm{Population wird immer grösser} \\
|\lambda_{dom}| < 1 & \, \textrm{Population wird immer kleiner} \\
|\lambda_{dom}| = 1 & \, \textrm{Population pendelt sich ein, bleibt gleich} \\
\end{array}
\right. $

\subsubsection{}

eig\_dom.m:
\begin{lstlisting}
function dom = eig_dom (M)
  LAM = eig(M);
  big = 0;
  dom = [];
  
  for i=1:(size(LAM)(1))
    cur = abs(LAM(i));
    if big > cur
      continue
    elseif big == cur
      dom(size(dom)(1)+1,1) = LAM(i);
    else
      dom = [LAM(i)];
      big = cur;
    endif
  endfor
endfunction
\end{lstlisting}

\subsection{Schutz für die Unechte Karettschildkröte 1}

\subsection{Schutz für die Unechte Karettschildkröte 2}

\subsubsection{}
\[
Jahr 1
  \begin{pmatrix}
    0 \\
    100 * u_1 \\
    0 \\
    0 \\
    0 \\
    0 \\
  \end{pmatrix}
Jahr 2
  \begin{pmatrix}
    0 \\
    100 * u_1 * u_2 \\
    0 \\
    0 \\
    0 \\
    0 \\
  \end{pmatrix}
Jahr 8
  \begin{pmatrix}
    0 \\
    100 * u_1 * u_2^7 \\
    0 \\
    0 \\
    0 \\
    0 \\
  \end{pmatrix}
\]

\subsubsection{}
\[
Jahr 1
  \begin{pmatrix}
    100 \\
    100 * u_1 \\
    0 \\
    0 \\
    0 \\
    0 \\
  \end{pmatrix}
Jahr 2
  \begin{pmatrix}
    100 \\
    100 * u_1 + 100 * u_1 * u_2\\
    0 \\
    0 \\
    0 \\
    0 \\
  \end{pmatrix}
Jahr 8
  \begin{pmatrix}
    100 \\
    100 * (u_1 + \sum_{k=1}^{6} u_2^k * u_1) \\
    100 * u_2^7 * u_1 \\
    0 \\
    0 \\
    0 \\
  \end{pmatrix}
\]

\subsubsection{}
\Large {
$$\frac{u_2^7 * u_1}{u_1 + \sum_{k=1}^{6} u_2^k * u_1} 
= \frac{u_2^7}{1 + \sum_{k=1}^{6} u_2^k}$$ \\
$$= \frac{u_2^7}{1 + \frac{1 - u_2^7}{1 - u_2}-1} =
 \frac{u_2^7 - u_2^8}{1-u_2^7} = 0.040459$$
}

\subsubsection{}
\Large {
$$\frac{u_k^{d_k} - u_k^{(d_k+1)}}{1-u_k^{d_k}}$$
}

\subsubsection{}
(u**d-u**(d+1))/(2-u**(d-1)-u)
\[
    \begin{pmatrix}
     0.67470 \\
     0.048593 \\
     0.014884  \\
     0.051833 \\
     0.89010 \\
     0.89010 \\
     0.00033237 \\
    \end{pmatrix}
\]


\subsection{}






\end{document}
